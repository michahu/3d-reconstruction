\documentclass[11pt]{extarticle}
\usepackage[utf8]{inputenc}
\usepackage{cite}

%-------------------- Formatting

\setlength{\parindent}{1cm}
\setlength{\parskip}{6pt}

%-------------------- Title

\title{Camera PCA for 3D Reconstruction \vspace{-1ex}}
\author{Justin Yan, Michael Hu, Julian Knodt}
\date{}
\begin{document}
\maketitle

%-------------------- Text

\section*{Introduction}

Structure from Motion(SfM) is the re-creation of 3D models from multiple images directed at the
same object from different perspectives. The process for reconstruction is three steps. First,
locate keypoints in some parts of an image. This is often done through tools like Sift, and is
not considered part of the SfM algorithm. Then, match keypoints across images with each other,
stitching images together in a way that will allow for 3D reconstruction. Then, from the matched
keypoints, derive some representation of a 3D model, generally a point cloud. One such pipeline
for this is Bundler\cite{2006Szeliski}, one of the original tools for SfM. The paper uses the
Notre Dame church as a reference, and used a sampling of images from the internet in order to
reconstruct a visualization of the church. For lesser known monuments and more common objects,
there will be a much less extensive data set available, and thus for such data sets a complete
and accurate reconstruction is not always possible. Thus, the question arises, how can more
images be added to an existing data set that will allow for a better reconstruction? In order to
answer this question, we hope to find a general approach for finding the general location and
perspective of an image that would best contribute to the reconstruction of small objects.
Nominally, the small objects assumption implies that there are not multiple different components
of the object that are able to be captured which share no components that are physically the
same. Given that we want to find orientations that would contribute the most to reconstruction,
we additionally seek to show that there exists important images in existing reconstructions which
a reconstruction without them would be significantly worse than with them. These important
images can then be used as training data in order to predict the best location for the next
camera.
\section*{Related Work}
\section*{Implementation}
\section*{TODO Other Sections}
% TODO
\section*{Conclusion}
\section*{Future Work}


\bibliographystyle{unsrt}
\nocite{*}
\bibliography{reconstruction.bib}

\end{document}
